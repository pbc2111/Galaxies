\documentclass[11pt]{article}
% \usepackage{phase1}
\usepackage{graphics,graphicx,color}
\usepackage[usenames,dvipsnames,svgnames,table]{xcolor}
\usepackage[margin=1in]{geometry}
\usepackage{fancyhdr}
\usepackage{wrapfig}
\usepackage{epsfig}
\usepackage{amssymb}
\usepackage{wrapfig}
\usepackage{setspace}
\usepackage[labelfont=bf]{caption}
%\usepackage{enumitem}
\usepackage{hyperref}
\newcolumntype{C}{>{$}c<{$}}
\def\Vhrulefill{\leavevmode\leaders\hrule height 0.7ex depth \dimexpr0.4pt-0.7ex\hfill\kern0pt}

\newcommand{\etal}{et.~al.}

\begin{document}

\begin{center}
\textbf{\large Homework \#1: Cosmology \& Conversions}\\
\textbf{\large SOLUTIONS}
\vspace{2mm}

{\sc AST386: Prof.\ Caitlin Casey}
\end{center}
\vspace{4mm}

\begin{enumerate}
\item No solution necessary! Congratulations.

\item Converting between flux, flux density, luminosity and magnitudes
  are critical skills to professional astronomers, and often prove
  confusing.  Here are a set of exercises to make you more adept at
  these conversions.  You will need to visit this page to download a
  series of files to complete this set of problems:\\
  \begin{center}
\vspace{-0.5cm}
  \begin{tabular}{|l|l|}
    \multicolumn{2}{c}{www.as.utexas.edu/$\sim$cmcasey/ast386/hw1tools/}\\
    \hline
    Template Stellar A0V Spectrum & spectrum\_A0V.txt \\
    \hline
    Subaru $g$-band filter profile & subaru\_g.txt \\
    \hline
    Subaru $r$-band filter profile & subaru\_r.txt \\
    \hline
    Subaru $i$-band filter profile & subaru\_i.txt \\
    \hline
    Subaru $z$-band filter profile & subaru\_z.txt \\
    \hline
    Subaru $Y$-band filter profile & subaru\_y.txt \\
    \hline
  \end{tabular}\\
  \end{center}
  The first column of every file is the wavelength (in \AA).  The
  second column of the template stellar spectrum is given in $S_{\nu}$
  units (erg\,s$^{-1}$\,cm$^{-2}$\,Hz$^{-1}$), and the second column
  of the filter profile is representative of the total system response
  T$_\lambda$, or the product of the filter response, detector quantum
  efficiency, instrument and telescope throughput and atomsphereic
  transmission.  In other words, T$_\lambda$ is the fraction of light
  that gets through at a given wavelength.
  \begin{enumerate}
    \item Plot the spectrum of the A0V star, along with a
      flat-spectrum source of flux density 3631\,Jy, overplotting and
      labeling each of these filters.  Make sure the plot is legible,
      and add labels/legends sufficient for me to fully understand
      what is plotted.  You will need to rescale the filter curves to
      be clearly visible on the plot.

{\color{gray} I first converted all of the wavelengths into $\mu\!m$
  from $\AA$, and plotted just the filter system response functions
  from g through y-bands.  The flux density scale for the Vega
  template is on a completely different scale than the system response
  (which ranges from 0 to 1), so I multiplied the Vega template by
  $2\times10^{19}$ to scale nicely against the filter profiles.  I
  overplot a flat spectrum source of 3631\,Jy (also multiplied by
  2$\times10^{19}$ as a dashed line.}

\begin{figure}[h!]
\centering
%\includegraphics[width=4in]{plot2a.pdf}
\end{figure}

    \item the AB magnitude system is defined such that a flat-spectrum
      source with flux density of 3631\,Jy has a measured flux density
      of (...drumroll...) 3631\,Jy in all filters, regardless of filter
      bandwidth/shape.  In the $AB$ magnitude system, such a source
      would have a magnitude of 0 in all bands.  If the template
      spectrum for the A0V star were to represent Vega, what is the
      magnitude of Vega in the AB system across these five filters?
      (This is the offset between Vega magnitudes and AB magnitudes
      for these filters.)

{\color{gray} We first compute the flux density of Vega in each of the
  five bands using the following equation from Tokunaga \&\ Vacca
  (2005):
\begin{equation}
F_{filter}=\frac{\int\lambda F_\lambda(\lambda)S(\lambda)d\lambda}{\int\lambda S(\lambda)d\lambda}
\end{equation}
This requires that we convert from F$_\nu$ units to F$_\lambda$ units
by multiplying by c/$\lambda^2$.  Once we have the flux density for
Vega in erg\,s$^{-1}$\,cm$^{-2}$\,\AA$^{-1}$, we can convert back to
$F_\nu$ units (or you could have stayed in F$_\nu$ units the entire
time, and calculate the magnitude of Vega using:
\begin{equation}
M = -2.5\log\Big(\frac{F_{\nu,vega}}{3631\,{\rm Jy}}\Big)
\end{equation}

The results are that Vega should have the following magnitudes in these filters: 
-0.099 (g), 0.157 (r), 0.399 (i), 0.542 (z), and 0.622 (y).
 } 

      \item What do you notice about the values of these offsets and
        the shape of the spectrum of Vega?

{\color{gray} Vega is brighter than the flat spectrum source in $g$
  band and fainter in all other bands.  So it follows that the
  magnitude for Vega is $<$0 in $g$ and positive in all other bands
  (and is fainter and fainter towards redder filters).}

      \item Plot the A0V spectrum in units of $\nu\!L_{\nu}$, assuming
        it is Vega and Vega is 7.68\,pc from us (note that
        $1\,pc=3.086\times10^{16}\,m$).  Overplot
        $\lambda\!L_{\lambda}$ on the same plot for comparison.  From
        this plot, what can you surmise about how intrinsically bright
        Vega is compared to the Sun? (this should give you some
        feeling of why people sometimes plot in units of
        $\nu\!L_{\nu}$ or $\lambda\!L_\lambda$.)

{\color{gray} If we convert the spectrum to $\nu L_\nu$ or $\lambda
  L_\lambda$ we get the same result: the two quantities are
  equivalent!  From this plot I would surmise that Vega is brightest
  at 4000\AA, at about $\sim$60L$_\odot$.  Indeed, the apparent
  luminosity of Vega is $\sim$57L$_\odot$. Note that it's actual
  luminosity is 40L$_\odot$, and the difference is due to the star's
  high rate of rotation and differential emergent flux as a function
  of line-of-sight with respect to the axis of rotation.  In any case,
  $\nu L_\nu$ is useful for inferring how luminous whatever object
  actually is bolometrically, and what wavelengths of light dominate
  that luminosity!}

\begin{figure}[h!]
\centering
%\includegraphics[width=4in]{plot2d.pdf}
\end{figure}


  \end{enumerate}

\item This problem will introduce the concept of stellar population
  synthesis (SPS) models by building up a basic understanding of
  stellar populations.  SPS models are critical to how we understand
  the stellar emission of high-redshift galaxies as integrated light
  sources, and so it is very important to understand how they are
  built.  This problem deals with bolometric quantities, and the next
  problem will introduce some of the mechanics of building SPS models
  with real templates.
  \begin{enumerate}
    \item The Salpeter IMF (Salpeter 1955) is parameterized:
      \begin{equation}
        \xi(\log\!m) = \frac{d(N/V)}{d\log\!m} = \frac{dn}{d\log\!m}\propto\!m^{-x}
      \end{equation}
      where $x=1.35$.  Using this distribution of stellar masses, plot
      the cumulative stellar mass fraction from high masses to low, in
      other words $f(>\!m)$ vs. $m$.  You can stop at the brown
      dwarf/hydrogen burning limit, $\sim$80\,M$_{\rm jup}$.  What is
      the average mass of a star drawn at random from this Salpeter
      distribution (i.e. the expectation value)?

{\color{gray} To compute the cumulative fraction of stellar mass
  function, I first convert $\xi(\log m)$ into $dn/dm$.  The stellar
  mass function that exists at a mass $m$ or above is then equal to:
  \begin{equation}
    f(>m) = \int_{m}^{\infty}\frac{dn}{dm}m\,dm = \int_{m}^{\infty}m^{-2.35}m\,dm = \frac{m^{-0.35}}{0.35}
  \end{equation}
  This is plotted here:
\begin{figure}[h!]
\centering
%\includegraphics[width=4in]{plot1a.pdf}
\end{figure}
The expectation value for a mass distribution $p(m)=dn/dm$ is defined:
\begin{equation}
E(m) = \frac{\int m\,p(m)\,dm}{\int p(m)\,dm}
\end{equation}
For this distribution integrated between a stellar mass of 80\,M$_{\rm
  Jup}$ and 100\,M$_\odot$, the expectation value is
0.27\,M$_{\odot}$.
 }
    \item The relationship between a star's luminosity and mass can be
      parameterized roughly as:
      \vspace{1mm}
      \begin{tabular}{CCCC}
        \frac{L}{L_\odot}\approx0.23\Big(\frac{M}{M_\odot}\Big)^{2.3} & 
        \frac{L}{L_\odot}\approx\Big(\frac{M}{M_\odot}\Big)^{4} & 
        \frac{L}{L_\odot}\approx1.5\Big(\frac{M}{M_\odot}\Big)^{3.5} & 
        \frac{L}{L_\odot}\approx3200\Big(\frac{M}{M_\odot}\Big) \\
        (M<0.43M_\odot) &
        (0.43M_\odot<M<2M_\odot) &
        (2M_\odot<M<20M_\odot) &
        (M>20M_\odot) \\
      \end{tabular}
      \vspace{1mm}
      What is the highest mass main sequence star you expect to live past
      100\,Myr? 500\,Myr? 1\,Gyr?

{\color{gray} Using the mass scale I defined for part (a) I generated
  a luminosity function according to the above scalings.  Here's a
  plot of stellar lifetimes on the main sequence by mass:

\begin{figure}[h!]
\centering
%\includegraphics[width=4in]{plot1b.pdf}
\end{figure}

Then we can read off this plot the masses of the stars equal to the
lifetimes given: 5.3\,M$_\odot$ (100\,Myr), 2.8\,M$_\odot$ (500\,Myr),
and 2.1\,M$_\odot$ (1\,Gyr).
%max mass for 100Myr lifetime:       5.3398988
% Expectation value for mass for 100Myr population=       0.22866642
%max mass for 500Myr lifetime:       2.8024189
% Expectation value for mass for 500Myr population=       0.21279091
%max mass for 1Gyr lifetime:       2.1258521
% Expectation value for mass for 1Gyr population=       0.20500890

}

    \item If you assume that a stellar population is aged 500\,Myr,
      what would be the average mass of a star drawn at random?  How
      does this differ from your answer to part (a), and how would it
      differ for a stellar population that is aged 1\,Gyr? (You can
      ignore evolved stars for the purposes of this problem even
      though we know they're... important.)

{\color{gray} Here we just use the expectation value for a mass given
  a distribution, as in part (a), but we truncate the stellar mass
  distribution at its maximum mass given an age, as determined in part
  (b).  The result is the following average masses: 0.23\,M$_\odot$,
  0.21\,M$_\odot$, and 0.21\,M$_\odot$.  In other words, small, and
  not changing a whole lot since the IMF is very bottom heavy!}

    \item What is the fractional contribution for stars of a given
      mass $m$ to the total light emitted by a given stellar
      population?  Plot this as a cumulative distribution, $f_{\rm
        L}(>m)$ as a function of $m$.  Hint: it would be wise to
      convert $\xi(\log\!m)$ to $dn/dl$ for this step.  What can you
      say about what types of stars dominate the light of any given
      stellar population?

{\color{gray} For this problem, we need to convert $\xi(\log m)$ to
  $dn/dl$, using the luminosity-mass relations given in part (b).
  This requires some piecewise integration!  Since we want to know
  what the contribution is {\it above} a certain mass limit, it would
  be good to start at the high end.  The cumulative contribution to
  luminosity will be:
\begin{equation}
f_L(>m) = \int_{l(m)}^{\infty} \frac{dn}{dl}\,dl = \int_{l(m)}^{l_{\rm max}}\frac{3200}{ln(10)}\Big(\frac{L}{3200}\Big)^{-x-1}\,dl = \frac{3200^{x+2}}{ln(10)}\Big(\frac{l_{\rm max}^{1-x}}{1-x}-\frac{l(m)^{1-x}}{1-x}\Big)
\end{equation}
Here $l(m)$ is the luminosity of a star of mass $m$ and $l_{\rm max}$
is the luminosity of a 100\,M$_\odot$ star.  We proceed to piece
together this function $f_{L}(>m)$ between each mass interval in a
similar manner, adding the total light contribution from the higher
mass intervals to given piecewise $f_{L}(>m)$.  Here is a plot of the
final function:}
\begin{figure}[h!]
\centering
%\includegraphics[width=4in]{plot1d.pdf}
\end{figure}

    \item Generate the same plot in part (d) but adjust it to
      represent a 500\,Myr-old and 1\,Gyr-old stellar population.
      Mark the three curves (including the 0-age curve) clearly.

{\color{gray} To plot this, we have to remove the light contribution
  from all main sequence stars that would have died by the given
  epoch.  Amazingly, $>$99\%\ of the light is from stars that have
  lifetimes shorter than 100\,Myr, and so this makes a big difference!
  Computationally, an aged version of the function $f_{L}$ is
  calculated by taking
  $f^\prime_{L}(>m,\tau)=\frac{f_{L}(>m)-f_L(>m_{\rm
      max})}{1-f_L(>m_{\rm max})}$.  We get:

\begin{figure}[h!]
\centering
%\includegraphics[width=4in]{plot1e.pdf}
\end{figure}


  Obviously the elephant in the room is the lack of accounting for
  evolved massive stars, which would add another layer of complexity
  to this since they are bright, but this exercise still hammers home
  the point that only the brightest of the brightest stars contribute
  to the bolometric luminosity of a stellar population. Sad!}
\end{enumerate}
  \item This problem builds on the previous problem, but now you are
    asked to build up a stellar population model spectrally.  To do
    this you will need to download and unpack the contents of this
    folder:
    {\tt http://www.as.utexas.edu/$\sim$cmcasey/ast386/hw2tools/}
%{\tt http://www.pas.rochester.edu/$\sim$emamajek/EEM\_dwarf\_UBVIJHK\_colors\_Teff.txt}
    In that folder there is a file named kurucz93.tar.gz and a file
    named EEM\_dwarf\_UBVIJHK\_colors\_Teff.txt.  The first is a
    directory of stellar atmosphere model spectra from Kurucz (1993)
    for a range of metallicities, effective temperatures and surface
    gravities.  The txt file should be used to map effective
    temperature and luminosity back to mass\footnote{The table quotes
      T$_{\rm eff}$ for stars between $0.1<M/M_\odot<19.6$ with a
      rough gridding, so you'll have to interpolate between these
      points to come up with good T$_{\rm eff}$ estimates for all mass
      points on the scale you used in problem 1.  At high masses you
      should interpolate towards a 100\,M$_\odot$ star having a
      45,000\,K temperature.}.  You can read the readme files in the
    Kurucz directories for more information, in addition to the
    annotations of the txt file.
  \begin{enumerate}
  \item Using the same mass range as in problem 1, plot $T_{\rm
    eff}$ against stellar mass and stellar luminosity $L$ vs stellar
    mass (by interpolating the values given in the reference txt
    file).

{\color{gray} To do this, I read in the data from the
  EEM\_dwarf\_UBVIJHK\_colors\_Teff.txt file, and plotted those that
  had Msun and Teff values.  I took the log of both, and I added one
  data point at 100\,M$_\odot$ and 45,000\,K, then I performed an
  interpolation in IDL between the data points.  Overplotted on both
  of these plots are the original points from the EEM file (as red
  circles) and my interpolated array between (black line).  I have
  also overplotted the analytic approximation to luminosity (as given
  in problem 1b) on the mass-luminosity plot for fun (in blue), so you can see
  how they compare.

\begin{figure}[h!]
\centering
%\includegraphics[width=3in]{plot2a1.pdf}
%\includegraphics[width=3in]{plot2a2.pdf}
\end{figure}

}


  \item Now that you know the effective temperature and luminosity
    across our entire mass range, you can make a composite spectrum
    for the stellar population as a whole.  For each value of your
    stellar mass grid, you should read in the appropriate Kurucz
    model, choosing the closest one in effective temperature.  For the
    gravity you can adopt: $T_{\rm eff}\ge41000\,K$ (column g50),
    $36000\ge T_{\rm eff}<41000\,K$ (column g45), $9000\ge T_{\rm
      eff}<36000\,K$ (column g40), and $T_{\rm eff}<9000\,K$ (column
    g45).  You'll want to add all of the spectra of the stars
    together, proportional to how many stars of each type are in the
    stellar population.  Plot the resulting stellar population
    spectrum in $\nu L_\nu$ units against wavelength.  Be sure to make
    sensible choices for your axes and think about whether or not it
    would be best to use a linear or a log scale to present your
    results.

{\color{gray} This was a lot of work!  First I read in all of the
  templates in the kp00 directory and made an array of arrays of the
  flux density (in $f_\lambda$, as indicated in the readme file) for
  each temperature given.  I also made an array of those
  representative temperatures.  Then I generated an interpolated
  spectrum for each mass point of my mass array by combining the
  spectra in the two nearest temperature bins that span the $T_{\rm
    eff}$ value for the given mass point shown in part (a). I scaled
  the flux for the template linearly with temperature between points.

  Then to get the total spectrum for the stellar population, I
  multiplied $\xi(\log m)\times\delta\log m$ by the flux array at that
  $\log m$ point to get the total contribution from stars of that
  spectral type.  Then I summed all of those contributions up over all
  masses to create a template spectrum for the whole stellar
  population.  The units of my spectrum are still proportional to
  $f_{\lambda}$, so to convert to something proportional to $\nu
  L_{\nu}$, I multiplied through by the wavelength in $\AA$ to get
  $\lambda f_\lambda$, which is $\propto\nu L_\nu$.  This is the
  resulting spectrum I obtain (left panel).  I choose a log scale to
  show the dynamic range and I hone in on the optical through
  mid-infrared portion of the spectrum: }

\begin{figure}[h!]
\centering
%\includegraphics[width=3in]{plot2b.pdf}
%\includegraphics[width=3in]{plot2c.pdf}
\end{figure}


  \item Now go back and split up this spectrum into the contribution
    from stars in different mass ranges.  You choose 3-4 mass ranges
    that you think convey the most interesting results.  State what
    mass ranges you assume clearly and label them on your plot.

{\color{gray} (Plotted above right) Looking at the luminosities of
  stars in different mass ranges in part (a) I decided to make five
  bins of the following mass cuts: $M/M_\odot<0.5$, $0.5<M/M_\odot<2$,
  $2<M/M_\odot<5$, $5<M/M_\odot<20$, and $M/M_\odot>20$. We can see
  that the contribution of low mass stars is really quite a bit lower
  than the contribution of high-mass stars.}

  \item Using your results from problem 1 now generate a spectrum of
    this stellar population after it has aged 500\,Myr, and dilineate
    the contributions from each of the mass ranges chosen in part
    (c).  Then, do the same for a 1\,Gyr age.

{\color{gray} To do this I follow a similar procedure as I used in
  problem 1, by noting the maximum main sequence stellar mass I expect
  to still be around at 500\,Myr or 1\,Gyr.  As noted there, these
  masses are 2.8\,M$_\odot$ and 2.1\,M$_\odot$ respectively.  This
  means the two highest mass ranges from part (c) no longer exist!
  (errr, well as evolved stars, which we're ignoring).  Here are the
  results of what's left, plotted as the total spectrum in black and
  the remaining stellar mass ranges colored appropriately.}

\begin{figure}[h!]
\centering
%\includegraphics[width=3in]{plot2d1.pdf}
%\\\includegraphics[width=3in]{plot2d2.pdf}
\end{figure}


  \item What differences and similarities do you notice across these
    stellar population models?  What types of stars dominate the
    spectrum over what wavelength range?

{\color{gray} The spectra are overall quite different!  The 0-age
  stellar population model is very blue and basically equal to the
  aggregate spectra of all of the massive OB stars.  As the population
  ages, the UV/blue light diminishes substantially.  I do notice that
  the SED shape at long wavelengths appears largely similar at all
  ages though, with the same intrinsic slope.  It seems like most of
  the light is dominated by the highest mass stars that exist at any
  given time.}

\end{enumerate}
\end{enumerate}


\end{document}
